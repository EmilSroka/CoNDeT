\documentclass{article}
\usepackage[utf8]{inputenc}
\usepackage[T1]{fontenc}
\usepackage[polish]{babel}
\usepackage{hyperref}
\def\code#1{\texttt{#1}}

\title{Dokumentacja biblioteki do wizualizacji CoNDeT}
\author{Emil Sroka, Hubert Miziołek, Miłosz Wrzesień}
\date{2021}

\begin{document}

\maketitle

\tableofcontents

\clearpage

\section{Historia dokumentacji}
\textbf{NOTE: do usunięcia w ostatecznej wersji}

\subsection{Wersja I.0}
\begin{itemize}
    \item Architektura –- Iteracja I
    \begin{itemize}
        \item Moduł prezentacji -- ogólne założenia
        \item Moduł prezentacji -- komponenty
        \item Wylistowanie pozostałych modułów
    \end{itemize}
\end{itemize}

\clearpage

\section{Wstęp}
\textbf{TODO: w nieokreślonej wersja}

\section{Architektura -- Iteracja I}

\paragraph{Założenia bazowe}

W celu uniknięcia kolizji nazw z innymi bibliotekami zostanie zastosowany wzorzec \code{namespace}. Globalny obiekt \code{CoNDeT} będzie przechowywał poszczególne moduły biblioteki.

\subsection{Moduł wspólny}

Moduł wspólny będzie znajdował się pod kluczem \code{core} w obiekcie biblioteki. Będzie on odpowiedzialny za funkcji oraz typy współdzielone między modułami.

\textbf{NOTE: dokumentacja modułu tworzona na bieżąco wraz z dodawaniem współdzielonych funkcjonalności}

\subsection{Moduł główny}

Moduł spinający pozostałe moduły.

\subsection{Moduł danych}

Moduł odpowiedzialny za obsługę danych. Będzie znajdował się pod kluczem \code{data} w obiekcie biblioteki.

\textbf{TODO: przechowywanie danych i wywoływanie procesu aktualizacji widoku -- w wersji II}

\textbf{TODO: wczytywanie i zapisywanie danych do pliku -- w wersji II lub III}

\textbf{TODO: interface do modyfikacji danych z poza modułu (np. z modułu UI w trybie edycji) -- w wersji III lub IV}

\subsection{Moduł prezentacji}

Moduł prezentacji będzie znajdował się pod kluczem \code{ui} w obiekcie biblioteki.

\subsubsection{Ogólne założenia}
\paragraph{Miejsce na stronie}
Interfejs użytkownika będzie generowany wewnątrz pierwszego podanego elementu \code{div} o podanym \code{id}. Jeżeli \code{id} nie zostanie podane, domyślnie będzie poszukiwany \code{div} o \code{id} równym \code{condet-canvas}.

\paragraph{Technologie}
Wizualizacja będzie bazowała na elementach HTML (tabele) oraz SVG (połączenia między tabelami). Operacje na elementach będą odbywały się pośrednio poprzez \hyperref[sssec:components]{komponenty opisane poniżej}.

\paragraph{Obiekt zarządzający}
Obiektem zarządzającym interfejsem użytkownika będzie \code{UserInterface}. Będzie on tworzył, aktualizował i usuwał komponenty.

\paragraph{Rozmieszczenie obiektów}
Obiekty będą rozmieszczone wewnątrz wybranego elementu \code{div}. Rozmiar powinien być ustalony przez osobę korzystającą z biblioteki. \code{UserInterface} nada elementowi style (\code{overflow: hidden;}) ukrywające zawartość wychodzącą poza \code{div} oraz będzie zapisywał offset. Komponenty będą miały przekazywany aktualny offset (jako parametr funkcji wchodzących w skład cyklu życia komponentu). Bazując na tym komponent powinien sam ustalić swoją pozycję. Umożliwi to przesuwanie widoku za pomocą mechanizmu drag \& drop.

\paragraph{Tryby}
Za pomocą wzorca stan (state design pattern) komponenty oraz obiekt \code{UserInterface} będą mieć różne zachowania zależnie od stanu. Zasada działania została \hyperref[p:eventsHandling]{opisana przy komponentach}. Pozwoli to na implementację dwóch trybów: wyświetlania oraz edycji.

\paragraph{Tryby wyświetlania}
Tryb wyświetlania pozwoli jedynie na wyświetlanie tabeli i połączeń między nimi.

\paragraph{Tryby edycji}
\textbf{TODO: w wersji III lub IV}

\subsubsection{Komponenty}
\label{sssec:components}

\paragraph{Bazowy komponent}
Bazową klasą po której będą dziedziczyć komponenty jest  \code{BaseComponent}. Wyodrębnia ona części wspólne wszystkich komponentów. Zawiera metody odpowiedzialne za cykl życia komponentu:

\begin{itemize}
    \item \code{init} -- metod wywoływana przy utworzeniu komponentu
    \item \code{update} -- metoda wywoływana przy zmianie stanu aplikacji (np. zawartość tabeli została zmieniona)
    \item \code{destroy} -- metoda wywoływana przy usuwaniu komponentu
\end{itemize}

\paragraph{Wymiary obiektu} Kolejną grupą metod są funkcja związane z położeniem i rozmiarem komponentu:

\begin{itemize}
    \item \code{containsPoint} -- metod sprawdza czy podany punkt zawiera się wewnątrz danego komponentu
    \item \code{getDimensions} -- metoda zwraca wymiary (szerokość i wysokość) komponentu
    \item \code{getPosistion} -- metoda zwraca pozycję (x i y) komponentu
\end{itemize}

\paragraph{Obsługa zdarzeń}
\label{p:eventsHandling}
Każdy komponent będzie również zawierał właściwość \code{state} (wzorzec stan). Konkretna implementacja stanu będzie odpowiadać stanowi w którym jest moduł prezentacji (np. stan prezentacji, stan edycji). Obiekt \code{State} będzie implementował metody związane z obsługą zdarzeń: \code{onKeyUp}, \code{onKeyDown}, \code{onMouseUp}, \code{onMouseDown}, oraz dwie własne metody wywoływane przy utworzeniu oraz zniszczeniu danej istanci: \code{onInit} oraz \code{onDestroy}. Metody te będą wywoływane przez bazowy komponent podczas wywołania metody \code{setState} służącej do zmiany stanu (wzorzec odwrócenie sterowania -- inversion of control design pattern). Do ustawienia subskrypcji na zdarzenia będzie służyć metoda \code{setupEventListiners} obecna w \code{BaseComponent}.

\paragraph{Zarządzanie dziećmi}
Każdy komponent będzie mógł być skomponowany z innych komponentów  (np. tabela może wykorzystywać komponent wiersza i nagłówka). Komponent bazowy będzie zawierał uniwersalną część logiki związaną z tym problemem. Właściwość \code{children} będzie tablicą dzieci. Do dodania nowego komponentu będzie służyć metoda \code{appendChild}, natomiast do usunięcia metoda \code{removeChild}.

\section{Format danych}
\subsection{Zastosowany Format}
Do zapisywania i wymiany danych w projekcie użyty zostanie format JSON ( JavaScript Object Notation ). Pozwala on zapis liczb, tekstu ( string ) , list, typów logicznych, obiektów oraz null'a. Odpowiednio zapisane i przygotowane dane w tym formacie uproszczą ich obróbkę przy ich wymianie z aplikacją.
\subsection{Struktura}
Wymiana danych będzie następowała poprzez wymianę pliku w formacie JSON. Każdy przesyłany plik zawiera w sobie listę danych, w której znajdują się obiekty reprezentujące pojedynczą tablicę decyzyjną. Każda tablica zawiera w sobie: 
\begin{enumerate}
    \item swoją nazwę - \textit{name}
    \item klasę określającą miejsce w hierarchii - \textit{class}
    \item obiekt zawierający koordynaty x i y określający położenie w oknie przeglądarki - \textit{coordinates}
    \item obiekt reprezentujący kolumny zawierający 2 listy z nazwami warunków oraz decyzji - \textit{columns}
    \item listę wierszy - \textit{rows}  zawierającą:
    \begin{enumerate}
        \item id danego wiersza - \textit{row\_id} 
        \item listę warunków w postaci tablicy dwuelementowej zawierającej id warunku i zawartość komórki - \textit{ conditions }
        \item listę decyzji w postaci tablicy dwuelementowej zawierającej id warunku i zawartość komórki  - \textit{ decisions }
        \item obiekt połączenie wskazujący na nazwę tablicy oraz id wiersza - \textit{ connection }
    \end{enumerate}
\end{enumerate}
\paragraph{Oszczędność pamięci} Zapisanie naszych tablic w ten sposób oszczędza ilość danych poprzez zlikwidowanie potrzeby wysyłania pustych pól tekstowych. Jest to możliwe dzięki nadaniu identyfikacji naszym warunkom oraz decyzjom.


\textbf{TODO: w wersji II}

\end{document}